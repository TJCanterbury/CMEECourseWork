\documentclass[12pt]{article}

\title{Autocorrelation in weather}
\author{Tristan Canterbury}
\date{}

\begin{document}
  \maketitle

  \begin{abstract}
    Here I analysed yearly temperature data for Key West, Florida for the 20th century. I found that for this location the temperature of a given years is significantly
    dependent on the temperature of the previous year. However the correlation is weak and so I'd argue that the predictive power of this model is poor. We also see
    a moderate and statistically significant increase in temperature over the 100 year period, which may be relevant for any meta analysis for measuring global temperature
    changes.

  \end{abstract}

  \section{Introduction}
    Whilst the weather can often seem eratic and unpredictable, especially day to day here in the UK, over greater spatial and temporal scales we tend to expect otherwise. 
    Here I test whether the weather in Key West, Florida has shown year to year continuity in it's mean ambient temperature, using an autocorrelation method. 


  \section{Materials \& Methods}
  By finding the correlation between 2 sample sequences of the temperature data, t0 and t1, of size n-1 where n is the number of years, and misalligning them 
  by one year I could find the correlation between each year and the year that followed. I then needed to test the significance of this correlation by finding 
  the probability that 2 randomized samples of the data of the same size would show a stronger correlation and using 100,000 permutations of this exercise to get 
  for a more accurate estimate.

  I also used cor.test to find the correlation between temperature and the year for the whole timescale just out of interest and for comparison.

  \section{Results}
  I found the correlation between successive yearly temperatures to be 0.3261697 and in this test only a 45 out of 100,000 permutations had a stronger correlation 
  giving a p-value of 4.5e-4. 
  A correlation of aprox. 0.3262 is not particularly strong as it means only approximately 11\% (r\textsuperscript{2} = 0.106) 
  of the variance in temperature is explainable by this correlation.
  However with a p-value of 4.5e-4 it is a statistically significant positive correlation.

  Meanwhile the pearson's product-moment correlation for year against temperature shows a moderate positive correlation of 0.533 and a p-value of 1.123e-8 which is very statistically significant.

  \section{Discussion}
  It seems clear that successive yearly temperatures in Key West are dependent upon one another to some extent. This means that a high mean temperature in one year predicts that the 
  next year may be similarly hot but the weakness of the correlation suggests that you should still expect large amounts of variance in temperature between years. So the predictive power 
  of this model is poor.

  Meanwhile we see that there is a stronger and more significant correlation over the longer time scale of a hundred years, showing that at greater temporal scales 
  we see a more predictable pattern in climate change, here showing a gradual increase in temperafture. This may be relevant for any meta analysis for measuring global temperature
  changes.

\end{document}
