\documentclass[12pt]{article}
\usepackage{graphicx}
\usepackage{amsmath}
\usepackage{fixltx2e}
\title{Temperature perfomance curves - better models needed for comparison}
\author{Tristan J Canterbury}
\date{30/11/2020}

\begin{document}
  \maketitle

  \begin{abstract}
  Here I use model fitting to investigate the validity of the Sharpe-Schoolfield model as a mechanistic model of temperature performance curves 
  and find that it outperforms 2 phenomological models, the Briere model and a quadratic model.
  I believe that the evidence points towards the Sharpe-Schoolfield model being a viable mechanistic model, giving credence
  to metabolic theory, on the grounds that it outperforms 
  these phenomological models and that the parameter estimates remain within biologically realistic bounds whilst doing so. 
  However, the poor performance of the Briere model, with it being a non-linear 
  model designed for similar data, tells me a better non linear model is required for comparison - because otherwise 
  the Sharpe-Schoolfield model may simply be benefitting from being a good non-linear model with coincidentally believable 
  parameter values.  

  \end{abstract}

  \section{An Introduction} 
  The ecological importance of temperature across species can be seen in their numerous physiological, morphological and 
  behavioural adaptations for temperature homeostasis. This and the potential lethality of relatively 
  high or low temperatures, forcing some species to even hibernate or aestivate within their 
  own niches, tells us maintaining a particular body temperature is 
  an inescapable hurdle for evolution. This leads us to ask by what mechanism does the 
  temperature of an organisms' environment effect it's biology and so have this impact on the 
  ecology and evolution of species. \\
  When we look at temperature performance curves, the relationship between biologicval traits and the temperature of an organisms' environment, we 
  see a complex picture. We do not simply see a normal distribution about an optimal temperature but an asymmetrical curve. It is suggested by metabolic theory 
  that these temperature response curves reflect the direct effects of temperature on the biochemical reactions relevant to the trait. The Sharpe-Schoolfield 
  model is a model built on this hypothesis and is the focus of this report. \\
  The Sharpe-Schoolfield Model has been adapted for non linear regression modelling and is built on the following explanation of temperature performance curves. 
  Chemical reactions have an activation energy, the threshold energy requirement for a collision between reactants to result in a reaction. This threshold can 
  be lowered through catalytic enzymes which are reusable and so should reduce the cost on the organism to release this energy from other 
  sources. However, the shapes of enzymes are not rigid and may denature at high temperatures to the point of inactivation or destruction. We see 
  a more gentle decrease in performance at lower temperatures as the probability of a high energy collision between reactant and enzyme decreases.\\
  In this report I investigate the performace of this model in comparison to 2 phenomological models, one being a generalised Briere model, which is non-linear; 
  and the other a simple quadratic model which is linear. \\

  \section{The Data}
  The dataset I use broadly holds performance data for photosynthesis and respiration at different temperatures. 
  Photosynthesis and respiration are well understood, vital life processes and both have some of their reactants 
  and products in the air making them relatively easy targets for study and modelling. There is considerable variance 
  in sample size and temperature range between IDs in the Dataset and so I examine the effect of this variation. 
  A large amount of IDs also have sample sizes below 6 
  and so these are excluded from the analysis as the larger sample sizes are required to fit a Sharpe-Schoolfield model. \\
  The data also comes from different species, coming from different habitats. We might expect differences in the peak performance 
  temperatures between species depending on the particular ecologies of the different species but this is not the focus 
  of this report.\\
 
  \section{The Models}
  I built 3 models for comparison, one being a simple linear quadratic model, one being a Briere model and one being a Sharpe-Schoolfield model.\\
  Linear models are generally the easiest models to generate because there is only one optimal set of coefficients that will model the data. 
  Where we save in computational time and human effort we lose biological meaning (and delayed gratification?). The coefficients we generate 
  do not have any biological meaning other than that they describe the shape of the data. However we can still use this to find the optimum 
  temperature for the performance of a trait and gain a better understanding for the shape of the performance curves which for many ecological 
  applications is probably enough. If the fit is terrible not so much and its ability to describe the data regardless of fit is also 
  fundamentally limited because it cannot show any changes in rates of rates of change and so, for instance, plateaus towards either end of a 
  curve and skewness will be hidden by this model.\\
  Briere Model \cite{Brierea} is a phenomological, non linear model that was designed to take the shape of a temperature response curve, 
  specifically that of temperature dependent development rates of arthropods. It takes 4 parameters, the minimum (T\textsubscript{0}) and maximum (T\textsubscript{m}) 
  feasible temperatures of the trait and B\textsubscript{0} as a normalisation constant and m for generalising the shape for other sorts of temperature
   curve. With 2 of the parameters in this model being more biologically meaning and the shape of the curve being more custom for temperature
    performance curves you would expect this to perform better than a quadratic model.\\

  \begin{equation}
    \begin{split}
        B = \left\{
                \begin{array}{ll}
                    0 & \quad T \leq T_0 \\
                    B_0 T (T-T_0) \sqrt{T_m-T} & \quad T_0 \leq T \leq T_m \\
                    0 & \quad T \geq T_m
                \end{array}
            \right.
        \end{split}
  \end{equation}

  Sharpe-Schoolfield model \cite{Schoolfield1981} is one such mechanistic model grounded in biochemistry and 
  it describes temperature response curves in the following way. Metabolic reactions have an activation energy 
  (defined as E in the model), the threshold energy requirement for a collision between reactants to result in 
  a reaction. This threshold is lowered through catalytic enzymes which are reusable and so should reduce the 
  cost on the organism to release this energy from other sources, The activation . However, the shapes of enzymes 
  are not rigid and so at higher temperatures they can denature to the point of inactivation, this temperature is 
  defined as T\textsubscript{h} and the energy is defined as E\textsubscript{h} which are the Temperature and energy at which 50\% of enzyme units 
  are inactive on the hotter side of the curve. We see a more gentle decrease in performance at lower temperatures 
  as the probability of a high energy collision between reactants and enzymes decreases, the model represents this
   with T\textsubscript{l} and E\textsubscript{l}, which are the energy and temperature when the performance is at half on the cooler side of the 
   curve. B\textsubscript{0} is then the real rate performance of the trait at a reference temperature of 283.15 kelvin, 
   intended to normalise the curves for comparison. \\
  
  \begin{equation}
    B = \frac{B_0 e^{\frac{-E}{k} (\frac{1}{T} - \frac{1}{283.15})}}
    { 1 + e^{\frac{E_l}{k} (\frac{1}{T_l} - \frac{1}{T})} + 
    e^{\frac{E_h}{k} (\frac{1}{T_h} - \frac{1}{T})}}
  \end{equation}

  To estimate the performance of fit of each model, while taking into account the number of parameters used 
  I used the AIC scores of the models. This tells us the information loss for a given model's estimates of the data
  and takes into account the simplicity of the model, measured in number of parameters. I then calculated the 
  proportional difference between the best model for each dataset and the other models. 

  \section{The Results}
  I found that Schoolfield comes out on top in AIC but the overlap in performance between the models is considerable. 
  In Figure 1 we see an example fit of ID 257 and you can see why Schoolfield 
  might be outperforming the other 2 models. It has a greater flexibility in shape due to the 
  additional parameters and in particular we can see it can do much sharper turns at the inflection point at 
  the peak of the curve. Figure 2 shows the probability that each model will give the best fit and we see that around 50\% of the time
  the Sharpe-Schoolfield Model wins which does not sound that impressive. Looking deeper into the data however reveals a possible reason, the 
  Quadratic model performs particularly well when the data 
  occupies a smaller range of temperatures, which we see in figure 3.\\
  Looking at Figure 4, the best fit parameter values for 
  the Sharpe-Schoolfield model are usually not so different from the 
  mathematically estimated values and appear biologically plausible.
  
  \begin{figure}[]
    \includegraphics[width=\textwidth]{../Results/plot110.pdf}
    Figure 1. Showing example of differences in fit between the models.
  \end{figure}

  \begin{figure}[]
    \includegraphics[width=\textwidth]{../Results/piechart.pdf}
    Figure 2. Percentage of IDs where each model gave the best performer by AIC score. 
  \end{figure}

  \begin{figure}[]
    \includegraphics[width=\textwidth]{../Results/Rangeplot.pdf}
    Figure 3. Comparing the effect of temperature range on AIC of the different models in terms of prob(min vs. i). 
    prob(min vs. i) is the probability of the model to minimise information loss as much as the the best of the 3 models.
  \end{figure}

  \begin{figure}[]
    \includegraphics[width=\textwidth]{../Results/Params.pdf}
    Figure 4. Both mathematicalle estimated and best fit Parameter valuesfor the Sharpe-Schoolfield model.
  \end{figure}

  \section{Discussion}

  My code starts with estimated parameter values based on mathematical models of the data and 
  so the benefit of more complete data for the Schoolfield model may well be that it allows more 
  accurate estimation of these parameters. In particular estimation of B\textsubscript{0} is 
  difficult when there is a lack of low temperature data. I used temperature range as a simplistic way of measuring
  the completeness of the data but categorising the data by whether it had sufficient low temperature data or high temperature data 
  and categorising the overall shape of the curves using polynomial models would be useful for understanding why exactly temperature 
  range has the effect it does in this data.\\
  The performance of the Briere model is surprisingly low for it being a non-linear model designed
  for the modelling of temperature performance curves. We do see improved performance for more complete data but overall 
  a linear quadratic model performs better. The quadratic model benefits from having fewer parameters and 
  we would expect it to better model straight lines and other irregular temperature performance curves, 
  which you would expect to find amongst incomplete datasets. Still the poor performance of the Briere model is unhelpful for evaluating the performance of the 
  Sharpe-Schoolfield model. It may be that when the data is good Sharpe-Schoolfield wins, when the data is bad quadratic wins, 
  and Briere sits somewhere in the middle in either scenario.\\
  Whether the lack of alternative mechanistic models is indicative of a lack of possible alternative mechanisms 
  or an ignorance of the existence of such models, is not known to me. Using model fitting to determine whether a 
  model is mechanistic seems counterintuitive, for it does not tell me why the model fits the data, only that 
  it is better than alternative models, so it is important to have as many models as possible to compare. The alternative models 
  in this case are also only expected to be phenomological
  and so were never probable alternatives from the beginning however they do serve their role as indicators of the 
  performance of this model, as we would expect these phenomological models to outperform a false mechanistic model.\\
  These results do point towards the Sharpe-Schoolfield model being a viable mechanistic model, giving credence
  to metabolic theory, on the grounds that it outperforms 
  these phenomological models and that the parameter estimates remain within biologically realistic bounds whilst doing so. 
  However, the poor performance of the Briere model, with it being a non-linear 
  model designed for similar data, tells me a better non linear model is required for comparison - because otherwise 
  the Sharpe-Schoolfield model may simply be benefitting from being a good non-linear model with coincidentally believable 
  parameter values. 


  \bibliographystyle{plain}
  \bibliography{CMEE}
\end{document}