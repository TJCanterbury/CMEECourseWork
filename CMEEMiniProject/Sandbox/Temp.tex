\documentclass[12pt]{article}

\title{How good are our mechanistic models of Temperature Performance curves for Photosynthesis and Respiration?}

\date{}

\begin{document}
  \maketitle

  \begin{abstract}
    Temperature performance curves are a growing interest amongst ecologists interested in abiotic factors. We know that life is built on chemistry and we know that
    much of our resources are spent on building enzymes to reduce the required activation energy for metabolic reactions. Photosynthesis and respiration both have reactants and products in the air, so these
    so their reaction rates are relatively easy to measure. The dataset I use here broadly holds performance data for photosynthesis and respiration when organisms are in environments of different temperature. 
    I measure the fit of different models to this data. I am interested in comparing the fit of the Schoolfield model (Schoolfield et al 1981) with phenomenological models such as simple quadratic and cubic models
    and the more specialised Briere model. 

  \end{abstract}

  \section{Introduction}
    It's Einstein time

  \section{Materials \& Methods}
  One of the most famous equations is:
  \begin{equation}
  E = mc^2
  \end{equation}
  This equation was first proposed by Einstein in 1905
  \cite{einstein1905does}.

  \section{Reesults}
  \section{Discussion}
  
  \bibliographystyle{plain}
  \bibliography{FirstBiblio}
\end{document}